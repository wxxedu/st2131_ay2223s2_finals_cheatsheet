\documentclass[11pt]{article}

\usepackage[margin=1cm]{geometry}
\geometry{a3paper}
\usepackage{amsmath,amssymb,amsthm}
\usepackage{graphicx}
\usepackage{multicol}
\usepackage{xcolor}

% No Indent 
\setlength\parindent{0pt}
% Line Spacing to 1 
\renewcommand{\baselinestretch}{1.0}
% Remove line spacing before and after sections 
\usepackage{titlesec}
% \titlespacing\section{0pt}{0pt plus 0pt minus 0pt}{0pt plus 0pt minus 0pt}
% \titlespacing\subsection{0pt}{0pt plus 0pt minus 0pt}{0pt plus 0pt minus 0pt}
% \titlespacing\subsubsection{0pt}{0pt plus 0pt minus 0pt}{0pt plus 0pt minus 0pt}

\renewcommand{\familydefault}{\sfdefault}

\AtBeginEnvironment{equation*}
{\setlength{\abovedisplayskip}{3pt}\setlength{\belowdisplayskip}{3pt}\color{purple}}
\AtBeginEnvironment{gather*}
{\setlength{\abovedisplayskip}{3pt}\setlength{\belowdisplayskip}{3pt}\color{purple}}
\AtBeginEnvironment{align*}
{\setlength{\abovedisplayskip}{3pt}\setlength{\belowdisplayskip}{3pt}\color{purple}}


% remove space before multicol 

\setlength{\multicolsep}{0pt}

% reduce list spacing
\usepackage{enumitem}
\setlist{nosep}

% Colors


% Color tweaks

\newcommand{\marker}[1]{{\color{magenta}{\textbf{#1}}}}

% set section color to violet 
\let\oldsection\section
\renewcommand{\section}[1]{\oldsection{\color{violet}{#1}}}
\renewcommand{\thesection}{\color{violet}{\arabic{section}}}

\let\oldsubsection\subsection
\renewcommand{\subsection}[1]{\oldsubsection{\color{purple}{#1}}}
\renewcommand{\thesubsection}{\color{purple}{\arabic{section}.\arabic{subsection}}}

% Change color of equation to blue 
\let\oldinlinemathstart\(
\renewcommand{\(}{\begingroup\color{purple}\oldinlinemathstart}
\let\oldinlinemathend\)
\renewcommand{\)}{\oldinlinemathend\endgroup}


\begin{document}
\begin{center}
\LARGE ST2131 Finals Cheatsheet by Wang Xiuxuan
\end{center}
\begin{multicols*}{3}
\section{Chapter 1}
% TODO
\section{Chapter 2}
% TODO
\section{Chapter 3}
% TODO
\section{Chapter 4}
\subsection{Random Variables}
\marker{Definition 4.1}: A \textbf{random variable}, \(X\), is a mapping from the
sample space to real numbers.
\subsection{Discrete Random Variables}

\marker{Definition 4.2}: A random variable is said to be \textbf{discrete} if 
range of \(X\) is either \textbf{fininite} or \textbf{countably infinite}.

\marker{Definition 4.3}: The \textbf{probability mass function} (pmf) of a
discrete random variable \(X\) is defined by
\begin{equation*}
p_X(x) = \begin{cases}P(X=x), & \text{if }x = x_1, x_2, \cdots \\ 0, & 
\text{otherwise}.\end{cases}
\end{equation*}
Note that \(\sum\limits_{i=1}^{\infty}p_X(x_i) = 1.\)

\marker{Definition 4.4}: The \textbf{cumulative distribution function} (cdf) of
a discrete random variable \(X\) is defined as 
\(F_X:\mathbb{R}\rightarrow \mathbb{R}T\) where \(F_X(x) = P(X\leq x)\), for 
\(x\in \mathbb{R}.\)

\subsection{Expected Value}

\marker{Definition 4.5}: If \(X\) is a discrete random variable having a 
probability mass function \(p_X\), the \textbf{expectation} or \textbf{expected 
value} of \(X\) is defined by 
\begin{equation*}
  E(X) = \sum\limits_{x}xp_X(x).
\end{equation*}

\marker{Tail Sum Formula}: For \textbf{nonnegative integer-valued} random 
variable \(X\), we have 
\begin{equation*}
E(X) = \sum\limits_{k=1}^{\infty}P(X \geq k) = 
\sum\limits_{k=0}^{\infty}P(X > k).
\end{equation*}

\subsection{Expectation of a Function of a Random Variable}

\marker{Proposition 4.1} If \(X\) is a \textbf{discrete random variable} that 
takes values \(x_i\), \(i \geq 1\), with respective probabilities \(p_X(x_i)\), 
then for any real value function \(g\), 
\begin{align*}
  E[g(X)] &= \sum_i g(x_i)p_X(x_i) \quad \text{or equivalently}\\
          &= \sum_x g(x)p_X(x).
\end{align*}

\marker{Corollary 4.2} Let \(a\) and \(b\) be constants, then 
\begin{equation*}
E[aX + b] = aE(X)+ b.
\end{equation*}

\subsection{Variance and Standard Deviation}

\marker{Definition 4.6} If \(X\) is a random variable with mean \(\mu\), then the 
\textbf{variance} of \(X\), denoted by \(\operatorname{Var}\left(X\right)\), is 
defined by 
\begin{equation*}
  \operatorname{Var}\left(X\right)=E[(X-\mu)^2].
\end{equation*}

\marker{Definition 4.7} The \textbf{standard deviation} of \(X\), denoted by 
\(\sigma_X\) or \(\operatorname{SD}\left(X\right)\), is defined by
\begin{equation*}
\sigma_X = \sqrt{\operatorname{Var}\left(X\right)}.
\end{equation*}

\marker{Scaling and shifting}: 
\begin{itemize}
  \item \(\operatorname{Var}\left(aX+b\right)=a^2\operatorname{Var}
    \left(X\right).\)
  \item \(\operatorname{SD}\left(aX+b\right)=|a|\operatorname{SD}
    \left(X\right).\)
\end{itemize}


\subsection{Discrete Random Variables Arising from Repeated Trials}

\marker{Bernoulli Random Variable}: \(\operatorname{Be}\left(p\right)\), define 
\begin{equation*}
  X = \begin{cases}1, &\text{if it is a success};\\ 
  0 &\text{if it is a failure.}\end{cases}
\end{equation*}
\begin{multicols}{2}
\begin{itemize}
\item \(P(X=1) = p\); 
\item \(P(X=0) = 1-p\);
\item \(E(X) = p\);
\item \(\operatorname{Var}\left(X\right)=p(1-p)\).
\end{itemize} 
\end{multicols}

\marker{Binomial Random Variable}: \(\operatorname{Bin}\left(n,p\right)\), define
\(X\) as \textbf{number of successes} in \(n\)
\(\operatorname{Bernoulli}\left(p\right)\) trials.

Therefore, \(X\) takes values \(0, 1, 2, \cdots, n\). 
\begin{itemize}
  \item \(P(X=k)=\binom{n}{k}p^k(1-p)^{n-k}\);
\end{itemize}
\begin{multicols}{2}
\begin{itemize}
  \item \(E(X)=np\);
  \item \(\operatorname{Var}\left(X\right)=np(1-p)\);
\end{itemize}
\end{multicols}

\marker{Geometric Random Variable}: \(\operatorname{Geom}\left(p\right)\), 
define \(X\) as \textbf{number of trials} until the first success in a 
sequence of independent \(\operatorname{Bernoulli}\left(p\right)\) trials.
\begin{itemize}
  \item \(P(X=k) = pq^{k-1}\);
\end{itemize}
\begin{multicols}{2}
\begin{itemize}
  \item \(E(X)=\dfrac{1}{p}\);
  \item \(\operatorname{Var}\left(X\right)=\dfrac{1 - p}{p^2}\);
\end{itemize}
\end{multicols}

\marker{Alternative Geometric Distribution}: Let \(X'\) be the 
\textbf{number of failures} in the \(\operatorname{Bernoulli}\left(p\right)\)
trials in order to obtain the first success. Here, 
\begin{equation*}
  X = X' + 1
\end{equation*}
\begin{itemize}
  \item \(P(X'=k) = pq^{k}\);
\end{itemize}
\begin{multicols}{2}
\begin{itemize}
  \item \(E(X')=\dfrac{1-p}{p}\);
  \item \(\operatorname{Var}\left(X'\right)=\dfrac{1 - p}{p^2}\);
\end{itemize}
\end{multicols}

\marker{Negative Binomial Random Variable}: \(\operatorname{NB}
\left(r,p\right)\), define \(X\) to be \textbf{number of
\(\operatorname{Bernoulli}\left(p\right)\) trials} required to obtain 
\textbf{$r$ successes}. For \(k \geq r\), we have:
\begin{itemize}
  \item \(P(X=k) = \binom{k-1}{r-1}p^r(1-p)^{k-r}\);
\end{itemize}
\begin{multicols}{2}
\begin{itemize}
  \item \(E(X)=\dfrac{r}{p}\);
  \item \(\operatorname{Var}\left(X\right)=\dfrac{r(1-p)}{p^2}\);
\end{itemize}
\end{multicols}

\marker{Remark}: \(\operatorname{Geom}\left(p\right)=
\operatorname{NB}\left(1,p\right).\)

\subsection{Poisson Random Variable}

\marker{Poisson}: A random variable \(X\) is said to have a \textbf{Poisson} 
distribution with parameter \(\lambda\) if \(X\) takes values 
\(0, 1, 2, \cdots\) with probabilities given as:
\begin{equation*}
  P(X = k) = \dfrac{e^{-\lambda}\lambda^k}{k!}, \quad \text{for }k=0, 1, 2, 
  \cdots
\end{equation*}
\begin{multicols}{2}
\begin{itemize}
  \item \(E(X)=\lambda\);
  \item \(\operatorname{Var}\left(X\right)=\lambda\).
\end{itemize}
\end{multicols}
When \(n\) is large and \(\lambda\) is moderate, we can use \textbf{poisson} to
approximate \textbf{binomial}:
\begin{equation*}
  \operatorname{Bin}\left(n,p\right) \approx \operatorname{Poisson}
  \left(np\right).
\end{equation*}
I.e. \(P(X = k)\approx e^{-\lambda}\dfrac{\lambda^k}{k!}\).

\marker{Examples obeying Poisson distribution}:
\begin{itemize}
  \item Number of misprints on a page;
  \item Number of people in a community living to 100 years;
  \item Number of wrong telephone numbers that are dialled in a day;
  \item Number of people entering a store on a given day;
  \item Number of particles emitted by a radioactive source.
\end{itemize}
These are approximately Poisson \textbf{because of the Poisson approximation 
to the binomial distribution.}
















\end{multicols*}
\end{document}
