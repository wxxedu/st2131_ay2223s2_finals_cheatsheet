\marker{Markov's Inequality} Let \(X\) be a
\textbf{nonnegative} random variable. For \(a > 0\), we have 
\[P(X \geq a)\leq \dfrac{E[X]}{a}.\]
\marker{Chebyshev's Inequality} Let \(X\) be a random variable
with mean \(\mu\) and variance \(\sigma^2\). For \(a > 0\), we have
\[P(|X-\mu| \geq a) \leq \dfrac{\sigma^2}{a^2}.\]

\marker{The Weak Law of Large Numbers} Let \(X_1, X_2, \cdots\) be
a sequence of \textbf{independent} and \textbf{identically} distributed 
random variables with common mean \(\mu\). Then, for any \(\epsilon > 0\), we
have
\[\lim\limits_{n\rightarrow
    \infty}P\left(\left|\dfrac{X_1+X_2+\cdots+X_n}{n}-\mu\right|\geq
\epsilon\right) = 0.\]

\marker{Central Limit theorem} Let \(X_1, X_2, \cdots\) be a sequence of
\textbf{independently} and \textbf{identically} 
distributed random variables, each having mean \(\mu\) and variance
\(\sigma^2\). Then the distribution of 
\[\dfrac{X_1+X_2+\cdots+X_n-n\mu}{\sigma\sqrt{n}}\]
tends to the standard norm as \(n \rightarrow \infty\). That is 
\begin{align*}
&\lim\limits_{n\rightarrow
\infty}P\left(\dfrac{X_1+X_2+\cdots+X_n-n\mu}{\sigma\sqrt{n}}\right) \\ 
  =&\dfrac{1}{\sqrt{2\pi}}\int_{x}^{-\infty}\exp(-t^2/2)\ \mathrm{d}t
\end{align*}
\marker{The Strong Law of Large Numbers} Let \(X_1, X_2, \cdots\)
be a sequence of independent and identically distributed random variables with
common mean \(\mu\). Then, with probability 1, we have
\[\lim\limits_{n\rightarrow\infty}\dfrac{X_1+X_2+\cdots+X_n}{n}=\mu.\]
\marker{One-sided Checbychev's Inequality} Let \(X\) be a
random variable with mean \(0\) and finite variance \(\sigma^2\). Then, for
\(a > 0\), we have
\[P(X \geq a) \leq \dfrac{\sigma^2}{\sigma^2+a^2}.\]
\marker{Jensen's Inequality} If \(g(x)\) is a \textbf{convex}
function, then 
\[g(E[X]) \leq E[g(X)]\]
provided that expectations exist and are finite.

\marker{Convex Functions} A function is convex if either of the following
equivalent conditions hold:
\begin{enumerate}
  \item for all \(0 \leq p \leq 1\) and for all \(x_1, x_2 \in R_X\), 
    \[g(px_1 + (1-p)x_2)\leq pg(x_1) + (1-p)g(x_2).\]
  \item differentiable function: convex of interval if and only if 
    \[g(x) \geq g(y) + g'(y)(x-y)\]
    for all \(x, y\) in the interval.
  \item A twice differentiable function is convex if and only if its second
    derivative is nonnegative.
\end{enumerate}
